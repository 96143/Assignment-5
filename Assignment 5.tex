\documentclass[journal,12pt,twocolumn]{IEEEtran}

\usepackage{setspace}
\usepackage{gensymb}
\singlespacing
\usepackage[cmex10]{amsmath}

\usepackage{amsthm}

\usepackage{multirow}
\usepackage{tikz}
\usetikzlibrary{matrix}
\tikzset{ 
table/.style={
  matrix of nodes,
  row sep=-\pgflinewidth,
  column sep=-\pgflinewidth,
  nodes={rectangle,text width=1em,align=center},
  text depth=1.25ex,
  text height=2.5ex,
  nodes in empty cells
},
row 1/.style={nodes={fill=green!10,text depth=0.4ex,text height=2ex}},
row 6/.style={nodes={text depth=0.4ex,text height=2ex}},
column 2/.style={nodes={text width=30ex,text height=2ex}},
column 3/.style={nodes={text width=18ex,text height=2ex}},
column 4/.style={nodes={text width=10ex,text height=2ex}},
column 1/.style={nodes={fill=green!10}},
}
\counterwithin{equation}{section}
\singlespacing

\usepackage{mathrsfs}
\usepackage{txfonts}
\usepackage{stfloats}
\usepackage{bm}
\usepackage{cite}
\usepackage{cases}
\usepackage{subfig}

\usepackage{longtable}
\usepackage{multirow}

\usepackage{enumitem}
\usepackage{mathtools}
\usepackage{steinmetz}
\usepackage{tikz}
\usepackage{circuitikz}
\usepackage{verbatim}
\usepackage{tfrupee}
\usepackage[breaklinks=true]{hyperref}
\usepackage{graphicx}
\usepackage{tkz-euclide}

\usetikzlibrary{calc,math}
\usepackage{listings}
    \usepackage{color}                                            %%
    \usepackage{array}                                            %%
    \usepackage{longtable}                                        %%
    \usepackage{calc}                                             %%
    \usepackage{multirow}                                         %%
    \usepackage{hhline}                                           %%
    \usepackage{ifthen}                                           %%
    \usepackage{lscape}     
\usepackage{multicol}
\usepackage{chngcntr}

\DeclareMathOperator*{\Res}{Res}

\renewcommand\thesection{\arabic{section}}
\renewcommand\thesubsection{\thesection.\arabic{subsection}}
\renewcommand\thesubsubsection{\thesubsection.\arabic{subsubsection}}

\renewcommand\thesectiondis{\arabic{section}}
\renewcommand\thesubsectiondis{\thesectiondis.\arabic{subsection}}
\renewcommand\thesubsubsectiondis{\thesubsectiondis.\arabic{subsubsection}}


\hyphenation{op-tical net-works semi-conduc-tor}
\def\inputGnumericTable{}                                 %%

\lstset{
%language=C,
frame=single, 
breaklines=true,
columns=fullflexible
}
\begin{document}


\newtheorem{theorem}{Theorem}[section]
\newtheorem{problem}{Problem}
\newtheorem{proposition}{Proposition}[section]
\newtheorem{lemma}{Lemma}[section]
\newtheorem{corollary}[theorem]{Corollary}
\newtheorem{example}{Example}[section]
\newtheorem{definition}[problem]{Definition}

\newcommand\Myperm[2][^n]{\prescript{#1\mkern-2.5mu}{}P_{#2}}
\newcommand\Mycomb[2][^n]{\prescript{#1\mkern-0.5mu}{}C_{#2}}
\newcommand{\BEQA}{\begin{eqnarray}}
\newcommand{\EEQA}{\end{eqnarray}}
\newcommand{\define}{\stackrel{\triangle}{=}}
\bibliographystyle{IEEEtran}
\raggedbottom
\setlength{\parindent}{0pt}
\providecommand{\mbf}{\mathbf}
\providecommand{\pr}[1]{\ensuremath{\Pr\left(#1\right)}}
\providecommand{\qfunc}[1]{\ensuremath{Q\left(#1\right)}}
\providecommand{\sbrak}[1]{\ensuremath{{}\left[#1\right]}}
\providecommand{\lsbrak}[1]{\ensuremath{{}\left[#1\right.}}
\providecommand{\rsbrak}[1]{\ensuremath{{}\left.#1\right]}}
\providecommand{\brak}[1]{\ensuremath{\left(#1\right)}}
\providecommand{\lbrak}[1]{\ensuremath{\left(#1\right.}}
\providecommand{\rbrak}[1]{\ensuremath{\left.#1\right)}}
\providecommand{\cbrak}[1]{\ensuremath{\left\{#1\right\}}}
\providecommand{\lcbrak}[1]{\ensuremath{\left\{#1\right.}}
\providecommand{\rcbrak}[1]{\ensuremath{\left.#1\right\}}}
\theoremstyle{remark}
\newtheorem{rem}{Remark}
\newcommand{\sgn}{\mathop{\mathrm{sgn}}}
\providecommand{\abs}[1]{\left\vert#1\right\vert}
\providecommand{\res}[1]{\Res\displaylimits_{#1}} 
\providecommand{\norm}[1]{\left\lVert#1\right\rVert}
%\providecommand{\norm}[1]{\lVert#1\rVert}
\providecommand{\mtx}[1]{\mathbf{#1}}
\providecommand{\mean}[1]{E\left[ #1 \right]}
\providecommand{\fourier}{\overset{\mathcal{F}}{ \rightleftharpoons}}
%\providecommand{\hilbert}{\overset{\mathcal{H}}{ \rightleftharpoons}}
\providecommand{\system}{\overset{\mathcal{H}}{ \longleftrightarrow}}
	%\newcommand{\solution}[2]{\textbf{Solution:}{#1}}
\newcommand{\solution}{\noindent \textbf{Solution: }}
\newcommand{\cosec}{\,\text{cosec}\,}
\providecommand{\dec}[2]{\ensuremath{\overset{#1}{\underset{#2}{\gtrless}}}}
\newcommand{\myvec}[1]{\ensuremath{\begin{pmatrix}#1\end{pmatrix}}}
\newcommand{\mydet}[1]{\ensuremath{\begin{vmatrix}#1\end{vmatrix}}}
\numberwithin{equation}{subsection}
\makeatletter
\@addtoreset{figure}{problem}
\makeatother
\let\StandardTheFigure\thefigure
\let\vec\mathbf
\renewcommand{\thefigure}{\theproblem}
\def\putbox#1#2#3{\makebox[0in][l]{\makebox[#1][l]{}\raisebox{\baselineskip}[0in][0in]{\raisebox{#2}[0in][0in]{#3}}}}
     \def\rightbox#1{\makebox[0in][r]{#1}}
     \def\centbox#1{\makebox[0in]{#1}}
     \def\topbox#1{\raisebox{-\baselineskip}[0in][0in]{#1}}
     \def\midbox#1{\raisebox{-0.5\baselineskip}[0in][0in]{#1}}
\vspace{3cm}
\title{AI5002: Assignment 5}
\author{Pradyumn Sharma\\ AI21MTECH02001}
\maketitle
\newpage
\bigskip
\renewcommand{\thefigure}{\theenumi}
\renewcommand{\thetable}{\theenumi}
%
latex codes from 
%
\begin{lstlisting}
https://github.com/96143/Assignment-3/tree/main
\end{lstlisting}
\section{Problem}
Let Z denote the sum of the numbers obtained
when two fair dice are rolled. Find the
variance and standard deviation of Z
\section{Solution}
Let X denotes the outcome of the first dice.Y is the outcome of the second dice.\\
\begin{equation}\label{eq:2.0.1}
\begin{split}
    X \in {1,2,3,4,5,6}\\
    Y \in {1,2,3,4,5,6}
\end{split}
\end{equation}
Now, \\
    Considering X and Y as independent.\\
    PMF of X:
    \begin{equation}\label{eq:2.0.2}
        p_X(n) = \pr{X=n} = \begin{cases}
    \frac{1}{6} & 1\geq n\leq 6 \\
    0 & otherwise \\
   \end{cases}
   \end{equation}
   PMF of Y:
    \begin{equation}\label{eq:2.0.3}
        p_Y(n) = \pr{Y=n} = \begin{cases}
    \frac{1}{6} & 1\geq n\leq 6 \\
    0 & otherwise \\
  \end{cases}
    \end{equation}
 Let Z be the sum of the outcomes of two die.
   Since,
   \begin{equation}\label{eq:2.0.4}
    Z = X + Y
   \end{equation}
  PMF of Z:
  \begin{equation}\label{eq:2.0.5}
        p_Z(n) = \pr{Z=n} = \begin{cases}
        0  & n < 1\\
    \frac{1}{6} \sum_{k=1}^{n-1} p_X(k)  & 1\geq n-1 \leq 6 \\
    \frac{1}{6} \sum_{k=n-6}^{6} p_X(k) & 1\geq n-6 \leq 6 \\
    0 & n > 12
  \end{cases}
    \end{equation}
    Using \eqref{eq:2.0.2} and \eqref{eq:2.0.3} in \eqref{eq:2.0.5} we get
    \begin{equation}\label{eq:2.0.6}
     p_Z(n) = \pr{Z=n} = \begin{cases}
        0  & n < 1\\
    \frac{n-1}{36}   & 2\geq n \leq 7 \\
    \frac{13-n}{36}  & 7\geq n \leq 12 \\
    0 & n > 12
    \end{cases}
    \end{equation}
    Now we have to find the variance.
\begin{equation}\label{eq:2.0.7}
     Var(X) = E[X^2] - (E[X])^2
\end{equation}
Finding E[X]
\begin{equation}\label{eq:2.0.8}
    E[X] = \sum_{k=1}^{n} k p_Z(k)
\end{equation}
    \begin{equation}\label{eq:2.0.9}
    \begin{split}
         E[X] = 0 +  \sum_{k=2}^{7} k\left(\frac{k-1}{36}\right) + \sum_{k=7}^{12} k\left(\frac{13-k}{36}\right) + 0\\
         E[X]  = 7
    \end{split}
    \end{equation}
   Finding $E[X^2]$
\begin{equation}\label{eq:2.0.10}
    E[X^2] = \sum_{k=1}^{n} k^2 p_Z(k)
\end{equation}
     \begin{equation}\label{eq:2.0.11}
    \begin{split}
         E[X^2] = 0 +  \sum_{k=2}^{7} k^2\left(\frac{k-1}{36}\right) + \sum_{k=7}^{12} k^2\left(\frac{13-k}{36}\right) + 0\\
         E[X^2]  = \frac{329}{6}
    \end{split}
    \end{equation}
    Using \eqref{eq:2.0.7} Variance is given by \\
\begin{equation*}
    Var(X) = \frac{329}{6} - (7)^2
\end{equation*}
\begin{equation*}
    Var(X) = \frac{35}{6} = 5.83
\end{equation*}
Standard Deviation is given by
\begin{equation}\label{eq:2.0.12}
    \sigma_x = \sqrt{Var(X)} 
\end{equation}
\begin{equation*}
    \sigma_x  = \sqrt{5.83} = 2.415
\end{equation*}
    

\end{document}

